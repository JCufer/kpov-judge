% $Header: /Users/joseph/Documents/LaTeX/beamer/solutions/generic-talks/generic-ornate-15min-45min.en.tex,v 90e850259b8b 2007/01/28 20:48:30 tantau $

\documentclass{beamer}

% This file is a solution template for:

% - Giving a talk on some subject.
% - The talk is between 15min and 45min long.
% - Style is ornate.

% Copyright 2004 by Till Tantau <tantau@users.sourceforge.net>.
%
% In principle, this file can be redistributed and/or modified under
% the terms of the GNU Public License, version 2.
%
% However, this file is supposed to be a template to be modified
% for your own needs. For this reason, if you use this file as a
% template and not specifically distribute it as part of a another
% package/program, I grant the extra permission to freely copy and
% modify this file as you see fit and even to delete this copyright
% notice. 


\mode<presentation>
{
  \usetheme[height=9mm]{ULFRI}
  % \usecolortheme{beaver}
  % or ...
  % \setbeamercovered{transparent}
  % or whatever (possibly just delete it)
}

% \usetheme[height=9mm]{ULFRI}

\usepackage[english]{babel}
% or whatever

\usepackage[utf8]{inputenc}
% or whatever
\usepackage{multirow}
\usepackage{times}
\usepackage[T1]{fontenc}
% Or whatever. Note that the encoding and the font should match. If T1
% does not look nice, try deleting the line with the fontenc.


\title[Kpov Judge] % (optional, use only with long paper titles)
{Strah in trepet prihodnjih generacij pri KPOV}

% \subtitle
% {A tale of RFIDs, finance and irritation} % (optional)

% (optional, use only with lots of authors)
\author[LUSY]
{Ga\v{s}per Fele-\v{Z}or\v{z}}

% - Use the \inst{?} command only if the authors have different
%   affiliation.

\institute[FRI]{
  Univerza v Ljubljani, Fakulteta za računalništvo in informatiko
  {\small polz@fri.uni-lj.si}
}

\date[FRI pedagoška delavnica] % (optional)
{}

\subject{KPOV Judge}
% This is only inserted into the PDF information catalog. Can be left
% out. 



% If you have a file called "university-logo-filename.xxx", where xxx
% is a graphic format that can be processed by latex or pdflatex,
% resp., then you can add a logo as follows:

% \pgfdeclareimage[height=0.5cm]{university-logo}{university-logo-filename}
% \logo{\pgfuseimage{university-logo}}



% Delete this, if you do not want the table of contents to pop up at
% the beginning of each subsection:
%\AtBeginSubsection[]
%{
%  \begin{frame}<beamer>{Outline}
%    \tableofcontents[currentsection,currentsubsection]
%  \end{frame}
%}


% If you wish to uncover everything in a step-wise fashion, uncomment
% the following command: 

%\beamerdefaultoverlayspecification{<+->}

\begin{document}

\begin{frame}{KPOV! Naravnost v obraz!}
  \titlepage
\end{frame}

\begin{frame}{Pregled}
  \tableofcontents
  % You might wish to add the option [pausesections]
\end{frame}


% Since this a solution template for a generic talk, very little can
% be said about how it should be structured. However, the talk length
% of between 15min and 45min and the theme suggest that you stick to
% the following rules:  

% - Exactly two or three sections (other than the summary).
% - At *most* three subsections per section.
% - Talk about 30s to 2min per frame. So there should be between about
%   15 and 30 frames, all told.
\section{Kaj je KPOV Judge} % -----------------------------------------------

\subsection{KPOV}

\begin{frame}{KPOV}
  \begin{itemize}
    \item Računalniške komunikacije 2
    \item Predpogoj za Spletne Tehnologije
    \item Postavi DHCP, TFTP, SNMP, RDATE, Video streaming...
    \item 90 študentov 2. letnika VSŠ, en asistent
  \end{itemize}
\end{frame}

\begin{frame}{Pot na FRI je tlakovana z dobrimi nameni}
  \begin{itemize}
    \item Dva tedna za prvi prototip s podporo za OpenCloud 
    \item Devet mesecev za približno delujočih nekaj nalog z navodili 
  \end{itemize}
\end{frame}

\begin{frame}{Čas izdelave}
  \includegraphics<1>[width=\textwidth,height=0.8\textheight,keepaspectratio]{figs/baby.jpg}
\end{frame}

\subsection{Uporabi že narejeno!}

\begin{frame}{Open Virtual Computer Lab}
  \includegraphics<1>[width=\textwidth,height=0.8\textheight,keepaspectratio]{figs/VCL.png}
  \begin{itemize}
    \item Postavi farmo strežnikov
    \item Napiši skripte za testiranje
  \end{itemize}
\end{frame}

\subsection{BYOD MOOC}
\begin{frame}{Običajna arhitektura}
  \includegraphics<1>[width=\textwidth,height=0.8\textheight,keepaspectratio]{figs/centralised-architecture.png}
\end{frame}

\begin{frame}{Prihranimo na strojni opremi!}
  \includegraphics<1>[width=\textwidth,height=0.8\textheight,keepaspectratio]{figs/kpov-architecture.png}
\end{frame}

\section{Kako se ga uporablja?}
\begin{frame}{Asistent}
  \begin{itemize}
    \item Navodila
    \item Seznam računalnikov / diskov, mrež
    \item Seznam parametrov
    \item Pripravljanje parametrov
    \item Pripravljanje diskov
    \item Preverjanje pri študentu
    \item Preverjanje na strežniku
  \end{itemize}
\end{frame}

\begin{frame}{Študent}
  \begin{itemize}
    \item Povleci virtualke
    \item Reši nalogo
    \item Poženi testni program
  \end{itemize}
\end{frame}

\begin{frame}{DEMO}
  \includegraphics<1>[width=\textwidth,height=0.8\textheight,keepaspectratio]{figs/fail.jpg}
\end{frame}

\section{Uporabljene tehnologije}
\begin{frame}{Uporabljene tehnologije}
  \begin{itemize}
    \item Python, Flask
    \item MongoDB
    \item libguestfs
    \item OpenStack (zadnjič deloval 2013)
  \end{itemize}
\end{frame}

\section{Rezultati uporabe}
\begin{frame}{Kako uporabiti študente}
  \begin{itemize}[<+->]
    \item Skupinsko delo (4 študentje / skupino)
    \item Jasno razdeljene naloge
    \item Če en ne opravi dela, padejo vsi
    \item Vsak je v dveh skupinah, vsak ima backup
    \item Brez točnih rokov
    \item Brez dobre dokumentacije
    \item Brez zadostnega obrtniškega znanja
  \end{itemize}
\end{frame}

\begin{frame}{Kako uporabiti študente}
  \begin{itemize}[<+->]
    \item V redu za howto, prevod
    \item Slabi za programiranje
  \end{itemize}
\end{frame}


\begin{frame}{Primer naloge}
  \begin{itemize}
    \item Preimenuj vse datoteke tako, da zamenjaš minuse s podčrtaji
    \item Napiši NAJKRAJŠI ukaz, ki premakne datoteke iz ADAHHD v KPOKNM
    \item Poišči datoteke, ki vsebujejo "mama"
    \item Izpiši zadnje vrstice, filtriraj syslog 
    \item Namesti paket "cowsay" in ga preizkusi :)
    \item Povleci datoteko s CURL
    \item Preštej vrstice v datoteki, rezultat spravi v okoljsko spremenljivko
    \item Preštej število vrstic v datoteki
  \end{itemize}
\end{frame}

\begin{frame}{Poizkusni zajčki}
  \begin{itemize}[<+->]
    \item Dva študenta pri RF, druga stopnja Bolognske
    \item En zaposleni, ne-asistent
    \item Trije asistentje
  \end{itemize}
\end{frame}

\begin{frame}{Koliko časa}
  \begin{itemize}
    \item En študent - dve uri, rešil pol
    \item Študentka - pet ur, rešila nič
    \item Zaposleni - pogledal, se ni resno lotil
    \item Dva asistenta - pogledala, po pol ure obupala
    \item En asistent rešil v slabih 2 urah
  \end{itemize}
\end{frame}

\section{Rad bi pomagal}

\begin{frame}{Dobri stari svn}
  svn co https://lusy.fri.uni-lj.si/svn/kpov-public
\end{frame}

\begin{frame}{Acknowledgements}
  \begin{itemize}
    \item Andrej Brodnik
    \item Andrej Tolič
    \item Študentje KPOV 2014/2015
  \end{itemize}
\end{frame}

\begin{frame}{Vprašanja}
\end{frame}
%\begin{frame}{Map of Slovenia}
  % - A title should summarize the slide in an understandable fashion
  %   for anyone how does not follow everything on the slide itself.
%  \includegraphics<1>[width=\textwidth,height=0.8\textheight,keepaspectratio]{slovenia-map.png}
%  \includegraphics<2>[width=\textwidth,height=0.8\textheight,keepaspectratio]{matkurja1.jpg}
  
%\end{frame}

\end{document}




